%% Generated by Sphinx.
\def\sphinxdocclass{report}
\documentclass[letterpaper,12pt,english]{sphinxmanual}
\ifdefined\pdfpxdimen
   \let\sphinxpxdimen\pdfpxdimen\else\newdimen\sphinxpxdimen
\fi \sphinxpxdimen=.75bp\relax
\usepackage[paperwidth=7.0in, paperheight=9.2in, bindingoffset=0.2in,left=1in,right=1in,top=1in,bottom=0.75in,footskip=.25in]{geometry}
\PassOptionsToPackage{warn}{textcomp}
\usepackage[utf8]{inputenc}
\ifdefined\DeclareUnicodeCharacter
% support both utf8 and utf8x syntaxes
\edef\sphinxdqmaybe{\ifdefined\DeclareUnicodeCharacterAsOptional\string"\fi}
  \DeclareUnicodeCharacter{\sphinxdqmaybe00A0}{\nobreakspace}
  \DeclareUnicodeCharacter{\sphinxdqmaybe2500}{\sphinxunichar{2500}}
  \DeclareUnicodeCharacter{\sphinxdqmaybe2502}{\sphinxunichar{2502}}
  \DeclareUnicodeCharacter{\sphinxdqmaybe2514}{\sphinxunichar{2514}}
  \DeclareUnicodeCharacter{\sphinxdqmaybe251C}{\sphinxunichar{251C}}
  \DeclareUnicodeCharacter{\sphinxdqmaybe2572}{\textbackslash}
\fi
\usepackage{cmap}
\usepackage[T1]{fontenc}
\usepackage{amsmath,amssymb,amstext}
\usepackage{babel}
\usepackage{times}
\usepackage[Bjarne]{fncychap}
\usepackage{sphinx}

\fvset{fontsize=\small}
\usepackage{geometry}

% Include hyperref last.
\usepackage{hyperref}
% Fix anchor placement for figures with captions.
\usepackage{hypcap}% it must be loaded after hyperref.
% Set up styles of URL: it should be placed after hyperref.
\urlstyle{same}

\addto\captionsenglish{\renewcommand{\figurename}{Fig.\@ }}
\makeatletter
\def\fnum@figure{\figurename\thefigure{}}
\makeatother
\addto\captionsenglish{\renewcommand{\tablename}{Table }}
\makeatletter
\def\fnum@table{\tablename\thetable{}}
\makeatother
\addto\captionsenglish{\renewcommand{\literalblockname}{Listing}}

\addto\captionsenglish{\renewcommand{\literalblockcontinuedname}{continued from previous page}}
\addto\captionsenglish{\renewcommand{\literalblockcontinuesname}{continues on next page}}
\addto\captionsenglish{\renewcommand{\sphinxnonalphabeticalgroupname}{Non-alphabetical}}
\addto\captionsenglish{\renewcommand{\sphinxsymbolsname}{Symbols}}
\addto\captionsenglish{\renewcommand{\sphinxnumbersname}{Numbers}}

\addto\extrasenglish{\def\pageautorefname{page}}

\setcounter{tocdepth}{1}

\usepackage{lastpage}
\usepackage{fancyhdr}

\pagestyle{fancy}
\fancyhf{}
\cfoot{\thepage\ of \pageref{LastPage}}


%% \usepackage[a5paper,bindingoffset=0.2in,%
%%             left=1in,right=1in,top=1in,bottom=1in,%
%%             footskip=.25in]{geometry}


\title{Mastering Django Admin}
\date{Sep 12, 2019}
\release{}
\author{ChillarAnand}
\newcommand{\sphinxlogo}{\vbox{}}
\renewcommand{\releasename}{}
\makeindex
\begin{document}

\pagestyle{empty}
\sphinxmaketitle
\pagestyle{plain}
\sphinxtableofcontents
\pagestyle{normal}
\phantomsection\label{\detokenize{index::doc}}



\chapter{Preface}
\label{\detokenize{preface:preface}}\label{\detokenize{preface::doc}}

\section{Why this book?}
\label{\detokenize{preface:why-this-book}}
blog posts


\section{Who should read this book?}
\label{\detokenize{preface:who-should-read-this-book}}
users


\section{Acknowlodgements}
\label{\detokenize{preface:acknowlodgements}}
Krace Kumar

Haris Ibrahim

Tim Graham,https://techytim.com/

Andrew Godwin, \sphinxurl{http://www.aeracode.org/}

Haki Banita

\sphinxurl{https://hakibenita.com/}


\chapter{The Million Dollar Admin}
\label{\detokenize{admin_million_dollars:the-million-dollar-admin}}\label{\detokenize{admin_million_dollars::doc}}
not to use for external applications

Not good ux

Django admin was first released in 2005 and it has gone through a lot of changes since then.

mental models

Revamp

reduce cost of maintainance and development

people come to django becase of admin

\sphinxurl{https://github.com/sshwsfc/xadmin}

\sphinxurl{https://jacobian.org/2016/may/26/so-you-want-a-new-admin/}


\chapter{Better Defaults}
\label{\detokenize{admin_better_defaults:better-defaults}}\label{\detokenize{admin_better_defaults::doc}}

\section{Set ordering for custom fields}
\label{\detokenize{admin_better_defaults:set-ordering-for-custom-fields}}
\begin{sphinxVerbatim}[commandchars=\\\{\}]
\PYG{k}{def} \PYG{n+nf}{number\PYGZus{}of\PYGZus{}orders}\PYG{p}{(}\PYG{n+nb+bp}{self}\PYG{p}{,} \PYG{n}{obj}\PYG{p}{)}\PYG{p}{:}
    \PYG{k}{return} \PYG{n}{obj}\PYG{o}{.}\PYG{n}{order\PYGZus{}\PYGZus{}count}
\PYG{n}{number\PYGZus{}of\PYGZus{}orders}\PYG{o}{.}\PYG{n}{admin\PYGZus{}order\PYGZus{}field} \PYG{o}{=} \PYG{l+s+s1}{\PYGZsq{}}\PYG{l+s+s1}{order\PYGZus{}\PYGZus{}count}\PYG{l+s+s1}{\PYGZsq{}}
\end{sphinxVerbatim}


\section{Auto completion}
\label{\detokenize{admin_better_defaults:auto-completion}}

\section{JSON Editor}
\label{\detokenize{admin_better_defaults:json-editor}}
Viewing and editing JSON field in admin interface will be very difficult with default text editor interface.

We can use 3rd packages like django-json-widget which provide JSON widget, with which viewing and editing JSON data becomes much intuitive.

\begin{sphinxVerbatim}[commandchars=\\\{\}]
\PYG{k+kn}{from} \PYG{n+nn}{django.contrib} \PYG{k+kn}{import} \PYG{n}{admin}
\PYG{k+kn}{from} \PYG{n+nn}{django.contrib.postgres} \PYG{k+kn}{import} \PYG{n}{fields}
\PYG{k+kn}{from} \PYG{n+nn}{django\PYGZus{}json\PYGZus{}widget.widgets} \PYG{k+kn}{import} \PYG{n}{JSONEditorWidget}
\PYG{k+kn}{from} \PYG{n+nn}{.models} \PYG{k+kn}{import} \PYG{n}{YourModel}


\PYG{n+nd}{@admin.register}\PYG{p}{(}\PYG{n}{YourModel}\PYG{p}{)}
\PYG{k}{class} \PYG{n+nc}{YourModelAdmin}\PYG{p}{(}\PYG{n}{admin}\PYG{o}{.}\PYG{n}{ModelAdmin}\PYG{p}{)}\PYG{p}{:}
    \PYG{n}{formfield\PYGZus{}overrides} \PYG{o}{=} \PYG{p}{\PYGZob{}}
        \PYG{n}{fields}\PYG{o}{.}\PYG{n}{JSONField}\PYG{p}{:} \PYG{p}{\PYGZob{}}\PYG{l+s+s1}{\PYGZsq{}}\PYG{l+s+s1}{widget}\PYG{l+s+s1}{\PYGZsq{}}\PYG{p}{:} \PYG{n}{JSONEditorWidget}\PYG{p}{\PYGZcb{}}\PYG{p}{,}
    \PYG{p}{\PYGZcb{}}
\end{sphinxVerbatim}

\sphinxurl{https://github.com/crucialfelix/django-ajax-selects}


\section{Sorting Models By Frequency}
\label{\detokenize{admin_better_defaults:sorting-models-by-frequency}}
\sphinxurl{https://github.com/mishbahr/django-modeladmin-reorder}


\section{Read-only fields}
\label{\detokenize{admin_better_defaults:read-only-fields}}\begin{quote}

readonly\_fields=(‘first’,)
\end{quote}

form help text

\sphinxurl{https://docs.djangoproject.com/en/dev/ref/models/fields/\#help-text}


\section{Customize Header/Title}
\label{\detokenize{admin_better_defaults:customize-header-title}}
\begin{sphinxVerbatim}[commandchars=\\\{\}]
\PYG{n}{admin}\PYG{o}{.}\PYG{n}{site}\PYG{o}{.}\PYG{n}{site\PYGZus{}header} \PYG{o}{=} \PYG{l+s+s1}{\PYGZsq{}}\PYG{l+s+s1}{My administration}\PYG{l+s+s1}{\PYGZsq{}}
\end{sphinxVerbatim}


\section{Plural names}
\label{\detokenize{admin_better_defaults:plural-names}}
\begin{sphinxVerbatim}[commandchars=\\\{\}]
\PYG{k}{class} \PYG{n+nc}{Category}\PYG{p}{(}\PYG{n}{models}\PYG{o}{.}\PYG{n}{Model}\PYG{p}{)}\PYG{p}{:}
    \PYG{k}{class} \PYG{n+nc}{Meta}\PYG{p}{:}
        \PYG{n}{verbose\PYGZus{}name\PYGZus{}plural} \PYG{o}{=} \PYG{l+s+s2}{\PYGZdq{}}\PYG{l+s+s2}{categories}\PYG{l+s+s2}{\PYGZdq{}}
\end{sphinxVerbatim}


\section{Disable links}
\label{\detokenize{admin_better_defaults:disable-links}}\begin{quote}

self.list\_display\_links = (None, )
\end{quote}


\section{Disable full count}
\label{\detokenize{admin_better_defaults:disable-full-count}}
\begin{sphinxVerbatim}[commandchars=\\\{\}]
\PYG{n}{show\PYGZus{}full\PYGZus{}result\PYGZus{}count} \PYG{o}{=} \PYG{n+nb+bp}{False}
\end{sphinxVerbatim}


\section{Allow editing in list view}
\label{\detokenize{admin_better_defaults:allow-editing-in-list-view}}
When a model is heavily used to update the content, it makes to sense to allow bulk edits on the models.

\begin{sphinxVerbatim}[commandchars=\\\{\}]
\PYG{k}{class} \PYG{n+nc}{BookAdmin}\PYG{p}{(}\PYG{n}{admin}\PYG{o}{.}\PYG{n}{ModelAdmin}\PYG{p}{)}\PYG{p}{:}
    \PYG{n}{list\PYGZus{}editable} \PYG{o}{=} \PYG{p}{(}\PYG{l+s+s1}{\PYGZsq{}}\PYG{l+s+s1}{author}\PYG{l+s+s1}{\PYGZsq{}}\PYG{p}{,}\PYG{p}{)}
\end{sphinxVerbatim}


\section{Fetch only required fields}
\label{\detokenize{admin_better_defaults:fetch-only-required-fields}}
When a model is registered in admin, django tries to fetch all the fields of the table in the query. If there are any joins involved, it fetch fields of the joined tables also. This will slow down the query when the table size is big or number of results per page is more.

To make queries faster, we can limit the queryset to fetch only required fields.

\begin{sphinxVerbatim}[commandchars=\\\{\}]
\PYG{k}{class} \PYG{n+nc}{BookAdmin}\PYG{p}{(}\PYG{n}{admin}\PYG{o}{.}\PYG{n}{ModelAdmin}\PYG{p}{)}\PYG{p}{:}
    \PYG{k}{def} \PYG{n+nf}{get\PYGZus{}queryset}\PYG{p}{(}\PYG{n+nb+bp}{self}\PYG{p}{,} \PYG{n}{request}\PYG{p}{)}\PYG{p}{:}
        \PYG{n}{qs} \PYG{o}{=} \PYG{n+nb}{super}\PYG{p}{(}\PYG{p}{)}\PYG{o}{.}\PYG{n}{get\PYGZus{}queryset}\PYG{p}{(}\PYG{n}{request}\PYG{p}{)}
        \PYG{n}{qs} \PYG{o}{=} \PYG{n}{qs}\PYG{o}{.}\PYG{n}{only}\PYG{p}{(}\PYG{l+s+s1}{\PYGZsq{}}\PYG{l+s+s1}{id}\PYG{l+s+s1}{\PYGZsq{}}\PYG{p}{,} \PYG{l+s+s1}{\PYGZsq{}}\PYG{l+s+s1}{name}\PYG{l+s+s1}{\PYGZsq{}}\PYG{p}{)}
        \PYG{k}{return} \PYG{n}{qs}


\PYG{n}{admin}\PYG{o}{.}\PYG{n}{site}\PYG{o}{.}\PYG{n}{register}\PYG{p}{(}\PYG{n}{Book}\PYG{p}{,} \PYG{n}{BookAdmin}\PYG{p}{)}
\end{sphinxVerbatim}


\section{Save as}
\label{\detokenize{admin_better_defaults:save-as}}

\section{radio fields}
\label{\detokenize{admin_better_defaults:radio-fields}}

\chapter{Auto Register All Models In Admin}
\label{\detokenize{admin_auto_register_models:auto-register-all-models-in-admin}}\label{\detokenize{admin_auto_register_models::doc}}

\section{Manual Registration}
\label{\detokenize{admin_auto_register_models:manual-registration}}
Inbuilt admin interface is one the most powerful \& popular feature of Django. Once we create the models, we need to register them with admin, so that it can read schema and populate interface for it.

Let us register Book model in the admin interface.

\begin{sphinxVerbatim}[commandchars=\\\{\}]
\PYG{c+c1}{\PYGZsh{} file: library/book/admin.py}

\PYG{k+kn}{from} \PYG{n+nn}{django.apps} \PYG{k+kn}{import} \PYG{n}{apps}

\PYG{k+kn}{from} \PYG{n+nn}{book.models} \PYG{k+kn}{import} \PYG{n}{Book}


\PYG{k}{class} \PYG{n+nc}{BookAdmin}\PYG{p}{(}\PYG{n}{admin}\PYG{o}{.}\PYG{n}{ModelAdmin}\PYG{p}{)}\PYG{p}{:}
    \PYG{n}{list\PYGZus{}display} \PYG{o}{=} \PYG{p}{(}\PYG{l+s+s1}{\PYGZsq{}}\PYG{l+s+s1}{id}\PYG{l+s+s1}{\PYGZsq{}}\PYG{p}{,} \PYG{l+s+s1}{\PYGZsq{}}\PYG{l+s+s1}{name}\PYG{l+s+s1}{\PYGZsq{}}\PYG{p}{,} \PYG{l+s+s1}{\PYGZsq{}}\PYG{l+s+s1}{author}\PYG{l+s+s1}{\PYGZsq{}}\PYG{p}{)}


\PYG{n}{admin}\PYG{o}{.}\PYG{n}{site}\PYG{o}{.}\PYG{n}{register}\PYG{p}{(}\PYG{n}{Book}\PYG{p}{,} \PYG{n}{BookAdmin}\PYG{p}{)}
\end{sphinxVerbatim}

Now, we can see the book model in admin.

\noindent{\hspace*{\fill}\sphinxincludegraphics{{admin-auto-register1}.png}\hspace*{\fill}}

If the django project has too many models to be registered in admin or if it has a legacy database where all tables need to be registered in admin, then adding all those models to admin becomes a tedious task.


\section{Auto Registration}
\label{\detokenize{admin_auto_register_models:auto-registration}}
To automate this process, we can programatically fetch all the models in the project and register them with admin. Also, we need to ignore models which are already registered with admin as django doesn’t allow regsitering same model twice.

\begin{sphinxVerbatim}[commandchars=\\\{\}]
\PYG{k+kn}{from} \PYG{n+nn}{django.apps} \PYG{k+kn}{import} \PYG{n}{apps}


\PYG{n}{models} \PYG{o}{=} \PYG{n}{apps}\PYG{o}{.}\PYG{n}{get\PYGZus{}models}\PYG{p}{(}\PYG{p}{)}

\PYG{k}{for} \PYG{n}{model} \PYG{o+ow}{in} \PYG{n}{models}\PYG{p}{:}
    \PYG{k}{try}\PYG{p}{:}
        \PYG{n}{admin}\PYG{o}{.}\PYG{n}{site}\PYG{o}{.}\PYG{n}{register}\PYG{p}{(}\PYG{n}{model}\PYG{p}{)}
    \PYG{k}{except} \PYG{n}{admin}\PYG{o}{.}\PYG{n}{sites}\PYG{o}{.}\PYG{n}{AlreadyRegistered}\PYG{p}{:}
        \PYG{k}{pass}
\end{sphinxVerbatim}

This code snippet should run after all \sphinxtitleref{admin.py} files are loaded so that auto registration happends after all manually added models are registered. Django provides AppConfig.ready() to perform any initialization tasks which can be used to hook this code.

\begin{sphinxVerbatim}[commandchars=\\\{\}]
\PYG{c+c1}{\PYGZsh{} file: library/book/apps.py}

\PYG{k+kn}{from} \PYG{n+nn}{django.apps} \PYG{k+kn}{import} \PYG{n}{apps}\PYG{p}{,} \PYG{n}{AppConfig}
\PYG{k+kn}{from} \PYG{n+nn}{django.contrib} \PYG{k+kn}{import} \PYG{n}{admin}


\PYG{k}{class} \PYG{n+nc}{BookAppConfig}\PYG{p}{(}\PYG{n}{AppConfig}\PYG{p}{)}\PYG{p}{:}

    \PYG{k}{def} \PYG{n+nf}{ready}\PYG{p}{(}\PYG{n+nb+bp}{self}\PYG{p}{)}\PYG{p}{:}
        \PYG{n}{models} \PYG{o}{=} \PYG{n}{apps}\PYG{o}{.}\PYG{n}{get\PYGZus{}models}\PYG{p}{(}\PYG{p}{)}
        \PYG{k}{for} \PYG{n}{model} \PYG{o+ow}{in} \PYG{n}{models}\PYG{p}{:}
            \PYG{k}{try}\PYG{p}{:}
                \PYG{n}{admin}\PYG{o}{.}\PYG{n}{site}\PYG{o}{.}\PYG{n}{register}\PYG{p}{(}\PYG{n}{model}\PYG{p}{)}
            \PYG{k}{except} \PYG{n}{admin}\PYG{o}{.}\PYG{n}{sites}\PYG{o}{.}\PYG{n}{AlreadyRegistered}\PYG{p}{:}
                \PYG{k}{pass}
\end{sphinxVerbatim}

In the admin, we can see manually registered models and automatically registered models. If we open admin page for any auto registered model, it will show something like this.

\noindent{\hspace*{\fill}\sphinxincludegraphics{{admin-auto-register2}.png}\hspace*{\fill}}

This view is not at all useful for the users who want to see the data. It will be more informative if we can show all the fields of the model in admin.


\section{Auto Registration With Fields}
\label{\detokenize{admin_auto_register_models:auto-registration-with-fields}}
To achieve that, we can create an admin class to populate model fields in \sphinxtitleref{list\_display}. While registering, we can use this admin class to register the model.

\begin{sphinxVerbatim}[commandchars=\\\{\}]
\PYG{k+kn}{from} \PYG{n+nn}{django.apps} \PYG{k+kn}{import} \PYG{n}{apps}\PYG{p}{,} \PYG{n}{AppConfig}
\PYG{k+kn}{from} \PYG{n+nn}{django.contrib} \PYG{k+kn}{import} \PYG{n}{admin}


\PYG{k}{class} \PYG{n+nc}{ListModelAdmin}\PYG{p}{(}\PYG{n}{admin}\PYG{o}{.}\PYG{n}{ModelAdmin}\PYG{p}{)}\PYG{p}{:}
    \PYG{k}{def} \PYG{n+nf+fm}{\PYGZus{}\PYGZus{}init\PYGZus{}\PYGZus{}}\PYG{p}{(}\PYG{n+nb+bp}{self}\PYG{p}{,} \PYG{n}{model}\PYG{p}{,} \PYG{n}{admin\PYGZus{}site}\PYG{p}{)}\PYG{p}{:}
        \PYG{n+nb+bp}{self}\PYG{o}{.}\PYG{n}{list\PYGZus{}display} \PYG{o}{=} \PYG{p}{[}\PYG{n}{field}\PYG{o}{.}\PYG{n}{name} \PYG{k}{for} \PYG{n}{field} \PYG{o+ow}{in} \PYG{n}{model}\PYG{o}{.}\PYG{n}{\PYGZus{}meta}\PYG{o}{.}\PYG{n}{fields}\PYG{p}{]}
        \PYG{n+nb}{super}\PYG{p}{(}\PYG{p}{)}\PYG{o}{.}\PYG{n+nf+fm}{\PYGZus{}\PYGZus{}init\PYGZus{}\PYGZus{}}\PYG{p}{(}\PYG{n}{model}\PYG{p}{,} \PYG{n}{admin\PYGZus{}site}\PYG{p}{)}


\PYG{k}{class} \PYG{n+nc}{BookAppConfig}\PYG{p}{(}\PYG{n}{AppConfig}\PYG{p}{)}\PYG{p}{:}

    \PYG{k}{def} \PYG{n+nf}{ready}\PYG{p}{(}\PYG{n+nb+bp}{self}\PYG{p}{)}\PYG{p}{:}
        \PYG{n}{models} \PYG{o}{=} \PYG{n}{apps}\PYG{o}{.}\PYG{n}{get\PYGZus{}models}\PYG{p}{(}\PYG{p}{)}
        \PYG{k}{for} \PYG{n}{model} \PYG{o+ow}{in} \PYG{n}{models}\PYG{p}{:}
            \PYG{k}{try}\PYG{p}{:}
                \PYG{n}{admin}\PYG{o}{.}\PYG{n}{site}\PYG{o}{.}\PYG{n}{register}\PYG{p}{(}\PYG{n}{model}\PYG{p}{,} \PYG{n}{ListModelAdmin}\PYG{p}{)}
            \PYG{k}{except} \PYG{n}{admin}\PYG{o}{.}\PYG{n}{sites}\PYG{o}{.}\PYG{n}{AlreadyRegistered}\PYG{p}{:}
                \PYG{k}{pass}
\end{sphinxVerbatim}

Now, if we look at Author admin page, it will be shown with all relevant fields.

\noindent{\hspace*{\fill}\sphinxincludegraphics{{admin-auto-register3}.png}\hspace*{\fill}}

Since we have auto registration in place, when a new model is added or columns are altered for existing models, admin interface will update accordingly without any code changes.


\chapter{Advanced Filtering}
\label{\detokenize{admin_filter:advanced-filtering}}\label{\detokenize{admin_filter::doc}}

\chapter{Allow ForeignKey Fields In Admin List Display}
\label{\detokenize{admin_list_display_foreignkey:allow-foreignkey-fields-in-admin-list-display}}\label{\detokenize{admin_list_display_foreignkey::doc}}
Django admin has \sphinxtitleref{ModelAdmin} class which provides options and functionality for the models in admin interface. It has options like \sphinxtitleref{list\_display}, \sphinxtitleref{list\_filter}, \sphinxtitleref{search\_fields} to specify fields for corresponding actions.

\sphinxtitleref{search\_fields}, \sphinxtitleref{list\_filter} and other options allow to include a ForeignKey or ManyToMany field with lookup API follow notation. For example, to search by book name in Bestselleradmin, we can specify \sphinxtitleref{book\_\_name} in search fields.

\begin{sphinxVerbatim}[commandchars=\\\{\}]
\PYG{k+kn}{from} \PYG{n+nn}{django.contrib} \PYG{k+kn}{import} \PYG{n}{admin}

\PYG{k+kn}{from} \PYG{n+nn}{book.models} \PYG{k+kn}{import} \PYG{n}{BestSeller}


\PYG{k}{class} \PYG{n+nc}{BestSellerAdmin}\PYG{p}{(}\PYG{n}{RelatedFieldAdmin}\PYG{p}{)}\PYG{p}{:}
    \PYG{n}{search\PYGZus{}fields} \PYG{o}{=} \PYG{p}{(}\PYG{l+s+s1}{\PYGZsq{}}\PYG{l+s+s1}{book\PYGZus{}\PYGZus{}name}\PYG{l+s+s1}{\PYGZsq{}}\PYG{p}{,} \PYG{p}{)}
    \PYG{n}{list\PYGZus{}display} \PYG{o}{=} \PYG{p}{(}\PYG{l+s+s1}{\PYGZsq{}}\PYG{l+s+s1}{id}\PYG{l+s+s1}{\PYGZsq{}}\PYG{p}{,} \PYG{l+s+s1}{\PYGZsq{}}\PYG{l+s+s1}{year}\PYG{l+s+s1}{\PYGZsq{}}\PYG{p}{,} \PYG{l+s+s1}{\PYGZsq{}}\PYG{l+s+s1}{rank}\PYG{l+s+s1}{\PYGZsq{}}\PYG{p}{,} \PYG{l+s+s1}{\PYGZsq{}}\PYG{l+s+s1}{book}\PYG{l+s+s1}{\PYGZsq{}}\PYG{p}{)}


\PYG{n}{admin}\PYG{o}{.}\PYG{n}{site}\PYG{o}{.}\PYG{n}{register}\PYG{p}{(}\PYG{n}{Bestseller}\PYG{p}{,} \PYG{n}{BestsellerAdmin}\PYG{p}{)}
\end{sphinxVerbatim}

However Django doesn’t allow the same follow notation in \sphinxtitleref{list\_display}. To include ForeignKey field or ManyToMany field in the list display, we have to write a custom method and add this method in list display.

\begin{sphinxVerbatim}[commandchars=\\\{\}]
\PYG{k+kn}{from} \PYG{n+nn}{django.contrib} \PYG{k+kn}{import} \PYG{n}{admin}

\PYG{k+kn}{from} \PYG{n+nn}{book.models} \PYG{k+kn}{import} \PYG{n}{BestSeller}


\PYG{k}{class} \PYG{n+nc}{BestSellerAdmin}\PYG{p}{(}\PYG{n}{RelatedFieldAdmin}\PYG{p}{)}\PYG{p}{:}
    \PYG{n}{list\PYGZus{}display} \PYG{o}{=} \PYG{p}{(}\PYG{l+s+s1}{\PYGZsq{}}\PYG{l+s+s1}{id}\PYG{l+s+s1}{\PYGZsq{}}\PYG{p}{,} \PYG{l+s+s1}{\PYGZsq{}}\PYG{l+s+s1}{rank}\PYG{l+s+s1}{\PYGZsq{}}\PYG{p}{,} \PYG{l+s+s1}{\PYGZsq{}}\PYG{l+s+s1}{year}\PYG{l+s+s1}{\PYGZsq{}}\PYG{p}{,} \PYG{l+s+s1}{\PYGZsq{}}\PYG{l+s+s1}{book}\PYG{l+s+s1}{\PYGZsq{}}\PYG{p}{,} \PYG{l+s+s1}{\PYGZsq{}}\PYG{l+s+s1}{author}\PYG{l+s+s1}{\PYGZsq{}}\PYG{p}{)}
    \PYG{n}{search\PYGZus{}fields} \PYG{o}{=} \PYG{p}{(}\PYG{l+s+s1}{\PYGZsq{}}\PYG{l+s+s1}{book\PYGZus{}\PYGZus{}name}\PYG{l+s+s1}{\PYGZsq{}}\PYG{p}{,} \PYG{p}{)}

    \PYG{k}{def} \PYG{n+nf}{author}\PYG{p}{(}\PYG{n+nb+bp}{self}\PYG{p}{,} \PYG{n}{obj}\PYG{p}{)}\PYG{p}{:}
        \PYG{k}{return} \PYG{n}{obj}\PYG{o}{.}\PYG{n}{book}\PYG{o}{.}\PYG{n}{author}
    \PYG{n}{author}\PYG{o}{.}\PYG{n}{description} \PYG{o}{=} \PYG{l+s+s1}{\PYGZsq{}}\PYG{l+s+s1}{Author}\PYG{l+s+s1}{\PYGZsq{}}


\PYG{n}{admin}\PYG{o}{.}\PYG{n}{site}\PYG{o}{.}\PYG{n}{register}\PYG{p}{(}\PYG{n}{Bestseller}\PYG{p}{,} \PYG{n}{BestsellerAdmin}\PYG{p}{)}
\end{sphinxVerbatim}

This way of adding foreignkeys in list\_display becomes tedious when there are lots of models with foreignkey fields.

We can write a custom admin class to dynamically set the methods as attributes so that we can use the ForeignKey fields in list\_display.

\begin{sphinxVerbatim}[commandchars=\\\{\}]
\PYG{k}{def} \PYG{n+nf}{get\PYGZus{}related\PYGZus{}field}\PYG{p}{(}\PYG{n}{name}\PYG{p}{,} \PYG{n}{admin\PYGZus{}order\PYGZus{}field}\PYG{o}{=}\PYG{n+nb+bp}{None}\PYG{p}{,} \PYG{n}{short\PYGZus{}description}\PYG{o}{=}\PYG{n+nb+bp}{None}\PYG{p}{)}\PYG{p}{:}
    \PYG{n}{related\PYGZus{}names} \PYG{o}{=} \PYG{n}{name}\PYG{o}{.}\PYG{n}{split}\PYG{p}{(}\PYG{l+s+s1}{\PYGZsq{}}\PYG{l+s+s1}{\PYGZus{}\PYGZus{}}\PYG{l+s+s1}{\PYGZsq{}}\PYG{p}{)}

    \PYG{k}{def} \PYG{n+nf}{dynamic\PYGZus{}attribute}\PYG{p}{(}\PYG{n}{obj}\PYG{p}{)}\PYG{p}{:}
        \PYG{k}{for} \PYG{n}{related\PYGZus{}name} \PYG{o+ow}{in} \PYG{n}{related\PYGZus{}names}\PYG{p}{:}
            \PYG{n}{obj} \PYG{o}{=} \PYG{n+nb}{getattr}\PYG{p}{(}\PYG{n}{obj}\PYG{p}{,} \PYG{n}{related\PYGZus{}name}\PYG{p}{)}
            \PYG{k}{return} \PYG{n}{obj}

    \PYG{n}{dynamic\PYGZus{}attribute}\PYG{o}{.}\PYG{n}{admin\PYGZus{}order\PYGZus{}field} \PYG{o}{=} \PYG{n}{admin\PYGZus{}order\PYGZus{}field} \PYG{o+ow}{or} \PYG{n}{name}
    \PYG{n}{dynamic\PYGZus{}attribute}\PYG{o}{.}\PYG{n}{short\PYGZus{}description} \PYG{o}{=} \PYG{n}{short\PYGZus{}description} \PYG{o+ow}{or} \PYG{n}{related\PYGZus{}names}\PYG{p}{[}\PYG{o}{\PYGZhy{}}\PYG{l+m+mi}{1}\PYG{p}{]}\PYG{o}{.}\PYG{n}{title}\PYG{p}{(}\PYG{p}{)}\PYG{o}{.}\PYG{n}{replace}\PYG{p}{(}\PYG{l+s+s1}{\PYGZsq{}}\PYG{l+s+s1}{\PYGZus{}}\PYG{l+s+s1}{\PYGZsq{}}\PYG{p}{,} \PYG{l+s+s1}{\PYGZsq{}}\PYG{l+s+s1}{ }\PYG{l+s+s1}{\PYGZsq{}}\PYG{p}{)}
    \PYG{k}{return} \PYG{n}{dynamic\PYGZus{}attribute}


\PYG{k}{class} \PYG{n+nc}{RelatedFieldAdmin}\PYG{p}{(}\PYG{n}{admin}\PYG{o}{.}\PYG{n}{ModelAdmin}\PYG{p}{)}\PYG{p}{:}
    \PYG{k}{def} \PYG{n+nf+fm}{\PYGZus{}\PYGZus{}getattr\PYGZus{}\PYGZus{}}\PYG{p}{(}\PYG{n+nb+bp}{self}\PYG{p}{,} \PYG{n}{attr}\PYG{p}{)}\PYG{p}{:}
        \PYG{k}{if} \PYG{l+s+s1}{\PYGZsq{}}\PYG{l+s+s1}{\PYGZus{}\PYGZus{}}\PYG{l+s+s1}{\PYGZsq{}} \PYG{o+ow}{in} \PYG{n}{attr}\PYG{p}{:}
            \PYG{k}{return} \PYG{n}{get\PYGZus{}related\PYGZus{}field}\PYG{p}{(}\PYG{n}{attr}\PYG{p}{)}

        \PYG{c+c1}{\PYGZsh{} not dynamic lookup, default behaviour}
        \PYG{k}{return} \PYG{n+nb+bp}{self}\PYG{o}{.}\PYG{n+nf+fm}{\PYGZus{}\PYGZus{}getattribute\PYGZus{}\PYGZus{}}\PYG{p}{(}\PYG{n}{attr}\PYG{p}{)}


\PYG{k}{class} \PYG{n+nc}{BestSellerAdmin}\PYG{p}{(}\PYG{n}{RelatedFieldAdmin}\PYG{p}{)}\PYG{p}{:}
    \PYG{n}{list\PYGZus{}display} \PYG{o}{=} \PYG{p}{(}\PYG{l+s+s1}{\PYGZsq{}}\PYG{l+s+s1}{id}\PYG{l+s+s1}{\PYGZsq{}}\PYG{p}{,} \PYG{l+s+s1}{\PYGZsq{}}\PYG{l+s+s1}{rank}\PYG{l+s+s1}{\PYGZsq{}}\PYG{p}{,} \PYG{l+s+s1}{\PYGZsq{}}\PYG{l+s+s1}{year}\PYG{l+s+s1}{\PYGZsq{}}\PYG{p}{,} \PYG{l+s+s1}{\PYGZsq{}}\PYG{l+s+s1}{book}\PYG{l+s+s1}{\PYGZsq{}}\PYG{p}{,} \PYG{l+s+s1}{\PYGZsq{}}\PYG{l+s+s1}{book\PYGZus{}\PYGZus{}author}\PYG{l+s+s1}{\PYGZsq{}}\PYG{p}{)}
\end{sphinxVerbatim}

By sublcassing RelatedFieldAdmin, we can directly use foreignkey fields in list display.

However, this will lead to N+1 problem. We will discuss more about this and how to fix this in orm optimizations chapter.


\chapter{Hyperlink Foreignkeys To Its Change View In Admin}
\label{\detokenize{admin_hyperlink_foreignkey:hyperlink-foreignkeys-to-its-change-view-in-admin}}\label{\detokenize{admin_hyperlink_foreignkey::doc}}
Consider Book model which has Author as foreignkey.

\begin{sphinxVerbatim}[commandchars=\\\{\}]
\PYG{k+kn}{from} \PYG{n+nn}{django.db} \PYG{k+kn}{import} \PYG{n}{models}


\PYG{k}{class} \PYG{n+nc}{Author}\PYG{p}{(}\PYG{n}{models}\PYG{o}{.}\PYG{n}{Model}\PYG{p}{)}\PYG{p}{:}
    \PYG{n}{name} \PYG{o}{=} \PYG{n}{models}\PYG{o}{.}\PYG{n}{CharField}\PYG{p}{(}\PYG{n}{max\PYGZus{}length}\PYG{o}{=}\PYG{l+m+mi}{100}\PYG{p}{)}

\PYG{k}{class} \PYG{n+nc}{Book}\PYG{p}{(}\PYG{n}{models}\PYG{o}{.}\PYG{n}{Model}\PYG{p}{)}\PYG{p}{:}
    \PYG{n}{title} \PYG{o}{=} \PYG{n}{models}\PYG{o}{.}\PYG{n}{CharField}\PYG{p}{(}\PYG{n}{max\PYGZus{}length}\PYG{o}{=}\PYG{l+m+mi}{100}\PYG{p}{)}
    \PYG{n}{author} \PYG{o}{=} \PYG{n}{models}\PYG{o}{.}\PYG{n}{ForeignKey}\PYG{p}{(}\PYG{n}{Author}\PYG{p}{)}
\end{sphinxVerbatim}

We can register these models with admin interface as follows.

\begin{sphinxVerbatim}[commandchars=\\\{\}]
\PYG{k+kn}{from} \PYG{n+nn}{django.contrib} \PYG{k+kn}{import} \PYG{n}{admin}

\PYG{k+kn}{from} \PYG{n+nn}{.models} \PYG{k+kn}{import} \PYG{n}{Author}\PYG{p}{,} \PYG{n}{Book}

\PYG{k}{class} \PYG{n+nc}{BookAdmin}\PYG{p}{(}\PYG{n}{admin}\PYG{o}{.}\PYG{n}{ModelAdmin}\PYG{p}{)}\PYG{p}{:}
    \PYG{n}{list\PYGZus{}display} \PYG{o}{=} \PYG{p}{(}\PYG{l+s+s1}{\PYGZsq{}}\PYG{l+s+s1}{name}\PYG{l+s+s1}{\PYGZsq{}}\PYG{p}{,} \PYG{l+s+s1}{\PYGZsq{}}\PYG{l+s+s1}{author}\PYG{l+s+s1}{\PYGZsq{}}\PYG{p}{,} \PYG{p}{)}

\PYG{n}{admin}\PYG{o}{.}\PYG{n}{site}\PYG{o}{.}\PYG{n}{register}\PYG{p}{(}\PYG{n}{Author}\PYG{p}{)}
\PYG{n}{admin}\PYG{o}{.}\PYG{n}{site}\PYG{o}{.}\PYG{n}{register}\PYG{p}{(}\PYG{n}{Book}\PYG{p}{,} \PYG{n}{BookAdmin}\PYG{p}{)}
\end{sphinxVerbatim}

Once they are registered, admin page shows Book model like this.

\noindent{\hspace*{\fill}\sphinxincludegraphics{{django-admin-fk-link-1}.png}\hspace*{\fill}}

While browsing books, we can see book name and author name. Here, book name field is liked to book change view. But author field is shown as plain text.

If we have to modify author name, we have to go back to authors admin page, search for relevant author and then change name.

This becomes tedious if users spend lot of time in admin for tasks like this. Instead, if author field is hyperlinked to author change view, we can directly go to that page and change the name.

Django provides an option to access admin views by its URL reversing system. For example, we can get change view of author model in book app using reverse(“admin:book\_author\_change”, args=id). Now we can use this url to hyperlink author field in book admin.

\begin{sphinxVerbatim}[commandchars=\\\{\}]
\PYG{k+kn}{from} \PYG{n+nn}{django.contrib} \PYG{k+kn}{import} \PYG{n}{admin}
\PYG{k+kn}{from} \PYG{n+nn}{django.utils.safestring} \PYG{k+kn}{import} \PYG{n}{mark\PYGZus{}safe}


\PYG{k}{class} \PYG{n+nc}{BookAdmin}\PYG{p}{(}\PYG{n}{admin}\PYG{o}{.}\PYG{n}{ModelAdmin}\PYG{p}{)}\PYG{p}{:}
    \PYG{n}{list\PYGZus{}display} \PYG{o}{=} \PYG{p}{(}\PYG{l+s+s1}{\PYGZsq{}}\PYG{l+s+s1}{name}\PYG{l+s+s1}{\PYGZsq{}}\PYG{p}{,} \PYG{l+s+s1}{\PYGZsq{}}\PYG{l+s+s1}{author\PYGZus{}link}\PYG{l+s+s1}{\PYGZsq{}}\PYG{p}{,} \PYG{p}{)}

    \PYG{k}{def} \PYG{n+nf}{author\PYGZus{}link}\PYG{p}{(}\PYG{n+nb+bp}{self}\PYG{p}{,} \PYG{n}{book}\PYG{p}{)}\PYG{p}{:}
        \PYG{n}{url} \PYG{o}{=} \PYG{n}{reverse}\PYG{p}{(}\PYG{l+s+s2}{\PYGZdq{}}\PYG{l+s+s2}{admin:book\PYGZus{}author\PYGZus{}change}\PYG{l+s+s2}{\PYGZdq{}}\PYG{p}{,} \PYG{n}{args}\PYG{o}{=}\PYG{p}{[}\PYG{n}{book}\PYG{o}{.}\PYG{n}{author}\PYG{o}{.}\PYG{n}{id}\PYG{p}{]}\PYG{p}{)}
        \PYG{n}{link} \PYG{o}{=} \PYG{l+s+s1}{\PYGZsq{}}\PYG{l+s+s1}{\PYGZlt{}a href=}\PYG{l+s+s1}{\PYGZdq{}}\PYG{l+s+si}{\PYGZpc{}s}\PYG{l+s+s1}{\PYGZdq{}}\PYG{l+s+s1}{\PYGZgt{}}\PYG{l+s+si}{\PYGZpc{}s}\PYG{l+s+s1}{\PYGZlt{}/a\PYGZgt{}}\PYG{l+s+s1}{\PYGZsq{}} \PYG{o}{\PYGZpc{}} \PYG{p}{(}\PYG{n}{url}\PYG{p}{,} \PYG{n}{book}\PYG{o}{.}\PYG{n}{author}\PYG{o}{.}\PYG{n}{name}\PYG{p}{)}
        \PYG{k}{return} \PYG{n}{mark\PYGZus{}safe}\PYG{p}{(}\PYG{n}{link}\PYG{p}{)}
    \PYG{n}{author\PYGZus{}link}\PYG{o}{.}\PYG{n}{short\PYGZus{}description} \PYG{o}{=} \PYG{l+s+s1}{\PYGZsq{}}\PYG{l+s+s1}{Author}\PYG{l+s+s1}{\PYGZsq{}}
\end{sphinxVerbatim}

Now in the book admin view, author field will be hyperlinked to its change view and we can visit just by clicking it.

Depending on requirements, we can link any field in django to other fields or add custom fields to improve productivity of users in admin.

Custom hyper links

\sphinxurl{https://docs.djangoproject.com/en/dev/ref/models/instances/\#get-absolute-url}


\chapter{Custom Admin Actions For Querysets \& Individual Objects}
\label{\detokenize{admin_custom_admin_actions:custom-admin-actions-for-querysets-individual-objects}}\label{\detokenize{admin_custom_admin_actions::doc}}

\section{Custom Actions On Querysets}
\label{\detokenize{admin_custom_admin_actions:custom-actions-on-querysets}}
Django provides admin actions which work on a queryset level. By default, django provides delete action in the admin.

In our books admin, we can select a bunch of books and delete them.

\noindent{\hspace*{\fill}\sphinxincludegraphics{{admin-custom-actions1}.png}\hspace*{\fill}}

Django provides an option to hook user defined actions to run additional actions on selected items. Let us write write a custom admin action to mark selected books as available.

\begin{sphinxVerbatim}[commandchars=\\\{\}]
\PYG{k}{class} \PYG{n+nc}{BookAdmin}\PYG{p}{(}\PYG{n}{admin}\PYG{o}{.}\PYG{n}{ModelAdmin}\PYG{p}{)}\PYG{p}{:}
    \PYG{n}{actions} \PYG{o}{=} \PYG{p}{(}\PYG{l+s+s1}{\PYGZsq{}}\PYG{l+s+s1}{make\PYGZus{}books\PYGZus{}available}\PYG{l+s+s1}{\PYGZsq{}}\PYG{p}{,}\PYG{p}{)}
    \PYG{n}{list\PYGZus{}display} \PYG{o}{=} \PYG{p}{(}\PYG{l+s+s1}{\PYGZsq{}}\PYG{l+s+s1}{id}\PYG{l+s+s1}{\PYGZsq{}}\PYG{p}{,} \PYG{l+s+s1}{\PYGZsq{}}\PYG{l+s+s1}{name}\PYG{l+s+s1}{\PYGZsq{}}\PYG{p}{,} \PYG{l+s+s1}{\PYGZsq{}}\PYG{l+s+s1}{author}\PYG{l+s+s1}{\PYGZsq{}}\PYG{p}{)}

    \PYG{k}{def} \PYG{n+nf}{make\PYGZus{}books\PYGZus{}available}\PYG{p}{(}\PYG{n+nb+bp}{self}\PYG{p}{,} \PYG{n}{modeladmin}\PYG{p}{,} \PYG{n}{request}\PYG{p}{,} \PYG{n}{queryset}\PYG{p}{)}\PYG{p}{:}
        \PYG{n}{queryset}\PYG{o}{.}\PYG{n}{update}\PYG{p}{(}\PYG{n}{is\PYGZus{}available}\PYG{o}{=}\PYG{n+nb+bp}{True}\PYG{p}{)}
    \PYG{n}{make\PYGZus{}books\PYGZus{}available}\PYG{o}{.}\PYG{n}{short\PYGZus{}description} \PYG{o}{=} \PYG{l+s+s2}{\PYGZdq{}}\PYG{l+s+s2}{Mark selected books as available}\PYG{l+s+s2}{\PYGZdq{}}
\end{sphinxVerbatim}

\noindent{\hspace*{\fill}\sphinxincludegraphics{{admin-custom-actions2}.png}\hspace*{\fill}}


\section{Custom Actions On Individual Objects}
\label{\detokenize{admin_custom_admin_actions:custom-actions-on-individual-objects}}
Custom admin actions are inefficient when taking action on an individual object. For example, to delete a single user, we need to follow these steps.
\begin{enumerate}
\def\theenumi{\arabic{enumi}}
\def\labelenumi{\theenumi .}
\makeatletter\def\p@enumii{\p@enumi \theenumi .}\makeatother
\item {} 
Select the checkbox of the object.

\item {} 
Click on the action dropdown.

\item {} 
Select “Delete selected” action.

\item {} 
Click on Go button.

\item {} 
Confirm that the objects needs to be deleted.

\end{enumerate}

Just to delete a single record, we have to perform 5 clicks. That’s too many clicks for a single action.

To simplify the process, we can have delete button at row level. This can be achieved by writing a function which will insert delete button for every record.

ModelAdmin instance provides a set of named URLs for CRUD operations. To get object url for a page, URL name will be \sphinxtitleref{\{\{ app\_label \}\}\_\{\{ model\_name \}\}\_\{\{ page \}\}}.

For example, to get delete URL of a book object, we can call \sphinxtitleref{reverse(“admin:book\_book\_delete”, args={[}book\_id{]})}. We can add a delete button with this link and add it to list\_display so that delete button is available for individual objects.

\begin{sphinxVerbatim}[commandchars=\\\{\}]
\PYG{k+kn}{from} \PYG{n+nn}{django.contrib} \PYG{k+kn}{import} \PYG{n}{admin}
\PYG{k+kn}{from} \PYG{n+nn}{django.utils.html} \PYG{k+kn}{import} \PYG{n}{format\PYGZus{}html}

\PYG{k+kn}{from} \PYG{n+nn}{book.models} \PYG{k+kn}{import} \PYG{n}{Book}


\PYG{k}{class} \PYG{n+nc}{BookAdmin}\PYG{p}{(}\PYG{n}{admin}\PYG{o}{.}\PYG{n}{ModelAdmin}\PYG{p}{)}\PYG{p}{:}
    \PYG{n}{list\PYGZus{}display} \PYG{o}{=} \PYG{p}{(}\PYG{l+s+s1}{\PYGZsq{}}\PYG{l+s+s1}{id}\PYG{l+s+s1}{\PYGZsq{}}\PYG{p}{,} \PYG{l+s+s1}{\PYGZsq{}}\PYG{l+s+s1}{name}\PYG{l+s+s1}{\PYGZsq{}}\PYG{p}{,} \PYG{l+s+s1}{\PYGZsq{}}\PYG{l+s+s1}{author}\PYG{l+s+s1}{\PYGZsq{}}\PYG{p}{,} \PYG{l+s+s1}{\PYGZsq{}}\PYG{l+s+s1}{is\PYGZus{}available}\PYG{l+s+s1}{\PYGZsq{}}\PYG{p}{,} \PYG{l+s+s1}{\PYGZsq{}}\PYG{l+s+s1}{delete}\PYG{l+s+s1}{\PYGZsq{}}\PYG{p}{)}

    \PYG{k}{def} \PYG{n+nf}{delete}\PYG{p}{(}\PYG{n+nb+bp}{self}\PYG{p}{,} \PYG{n}{obj}\PYG{p}{)}\PYG{p}{:}
        \PYG{n}{view\PYGZus{}name} \PYG{o}{=} \PYG{l+s+s2}{\PYGZdq{}}\PYG{l+s+s2}{admin:\PYGZob{}\PYGZcb{}\PYGZus{}\PYGZob{}\PYGZcb{}\PYGZus{}delete}\PYG{l+s+s2}{\PYGZdq{}}\PYG{o}{.}\PYG{n}{format}\PYG{p}{(}\PYG{n}{obj}\PYG{o}{.}\PYG{n}{\PYGZus{}meta}\PYG{o}{.}\PYG{n}{app\PYGZus{}label}\PYG{p}{,} \PYG{n}{obj}\PYG{o}{.}\PYG{n}{\PYGZus{}meta}\PYG{o}{.}\PYG{n}{model\PYGZus{}name}\PYG{p}{)}
        \PYG{n}{link} \PYG{o}{=} \PYG{n}{reverse}\PYG{p}{(}\PYG{n}{view\PYGZus{}name}\PYG{p}{,} \PYG{n}{args}\PYG{o}{=}\PYG{p}{[}\PYG{n}{book}\PYG{o}{.}\PYG{n}{pk}\PYG{p}{]}\PYG{p}{)}
        \PYG{n}{html} \PYG{o}{=} \PYG{l+s+s1}{\PYGZsq{}}\PYG{l+s+s1}{\PYGZlt{}input type=}\PYG{l+s+s1}{\PYGZdq{}}\PYG{l+s+s1}{button}\PYG{l+s+s1}{\PYGZdq{}}\PYG{l+s+s1}{ onclick=}\PYG{l+s+s1}{\PYGZdq{}}\PYG{l+s+s1}{location.href=}\PYG{l+s+se}{\PYGZbs{}\PYGZsq{}}\PYG{l+s+s1}{\PYGZob{}\PYGZcb{}}\PYG{l+s+se}{\PYGZbs{}\PYGZsq{}}\PYG{l+s+s1}{\PYGZdq{}}\PYG{l+s+s1}{ value=}\PYG{l+s+s1}{\PYGZdq{}}\PYG{l+s+s1}{Delete}\PYG{l+s+s1}{\PYGZdq{}}\PYG{l+s+s1}{ /\PYGZgt{}}\PYG{l+s+s1}{\PYGZsq{}}\PYG{o}{.}\PYG{n}{format}\PYG{p}{(}\PYG{n}{link}\PYG{p}{)}
        \PYG{k}{return} \PYG{n}{format\PYGZus{}html}\PYG{p}{(}\PYG{n}{html}\PYG{p}{)}
\end{sphinxVerbatim}

Now in the admin interface, we have delete button for individual objects.

\noindent{\hspace*{\fill}\sphinxincludegraphics{{admin-custom-actions3}.png}\hspace*{\fill}}

To delete an object, just click on delete button and then confirm to delete it. Now, we are deleting objects with just 2 clicks.

In the above example, we have used an inbuilt model admin delete view. We can also write custom view and link those views for custom actions on individual objects. For example, we can add a button which will mark the book status to available.

In this chapter, we have seen how to write custom admin actions which work on single item as well as bulk items.


\chapter{Securing Django Admin}
\label{\detokenize{admin_secure:securing-django-admin}}\label{\detokenize{admin_secure::doc}}

\section{Securing Server}
\label{\detokenize{admin_secure:securing-server}}

\subsection{VPN}
\label{\detokenize{admin_secure:vpn}}

\subsection{FIREWALL}
\label{\detokenize{admin_secure:firewall}}
80/443


\subsection{https}
\label{\detokenize{admin_secure:https}}

\section{Securing Django}
\label{\detokenize{admin_secure:securing-django}}
python manage.py check \textendash{}deploy


\subsection{allowed hosts}
\label{\detokenize{admin_secure:allowed-hosts}}

\subsection{Disable DEBUG}
\label{\detokenize{admin_secure:disable-debug}}

\subsection{Change default url}
\label{\detokenize{admin_secure:change-default-url}}

\subsection{Ensuring proper ACL}
\label{\detokenize{admin_secure:ensuring-proper-acl}}

\subsection{Honeypot}
\label{\detokenize{admin_secure:honeypot}}
\sphinxurl{https://github.com/dmpayton/django-admin-honeypot}


\subsection{2FA}
\label{\detokenize{admin_secure:fa}}
\sphinxurl{https://github.com/Bouke/django-two-factor-auth}


\subsection{ENVironment}
\label{\detokenize{admin_secure:environment}}
\sphinxurl{https://github.com/dizballanze/django-admin-env-notice}


\chapter{final words}
\label{\detokenize{final:final-words}}\label{\detokenize{final::doc}}
Think about workflows.

Don’t waster too much time.



\renewcommand{\indexname}{Index}
\printindex
\end{document}